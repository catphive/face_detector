\documentclass{article}
\title{Final Project: Face detector}
\author{Brendan Miller}
\date{March 12 2012}
\usepackage{multirow,amsmath}
\usepackage{graphicx}
\usepackage{fancyvrb}
\usepackage{color}
\usepackage{amsfonts}
\usepackage{hyperref}
\begin{document}
\maketitle

\begin{abstract}
A face detector implemented via a simplified Viola Jones algorithm.
\end{abstract}

\section{Project Overview}

For this project I implemented an binary image classifier that selects
faces from non-faces. This is based on the work of Paul Viola and
Michael Jones\cite{violajones2001}.

In the original Viola Jones paper, features were extracted from images
by first generating what they termed an integrated image. From the
integrated image, a large number of real valued features, described more fully in
the next section, could be computed quickly.

Given a large set of real valued features, a weak learner is
constructed by selecting a single feature and a threshhold value for
the feature that partitions example images into face and non-face
classes. This weak learner is conceptually similar to a decision tree
stump, although the metric used to find the optimal feature and
threashhold is different.

The weak learner is discussed more fully in section \ref{sec:weaklearner}.

Weak learners are combined with boosting (see section
\ref{sec:boosting}) in order to increase accuracy.

Finally, in the original Viola Jones paper, boosted classifiers were
combined in a cascade. The cascade was designed to reduce the
incendence of false positives, and to increase the performance of the
final classifer and possibly of the training phase. For my project I
have ommitted the cascade, as I said I would in my original project
proposal.

Even without the cascade, I've managed to achieve very high accuracy,
upwards of 99\%\footnote{Note this is on a validation set selected
  from the images provided with the project. I would expect accuracy
  to drop on images from a less carefully curated source.}, using a
boosted classifier over the features described in the original Viola
Jones paper.

\section{Feature Extraction}

\section{Weak Learner}
\label{sec:weaklearner}

\section{Boosted Classifier}
\label{sec:boosting}

\section{Experimental Results}

\begin{thebibliography}{9}

\bibitem{violajones2001}
  Paul Viola, Michael Jones\\
  Rapid Object Detection using a Boosted Cascade of Simple Features\\
  \url{http://research.microsoft.com/en-us/um/people/viola/Pubs/Detect/violaJones_CVPR2001.pdf}

\end{thebibliography}

\end{document}